\documentclass[ignorenonframetext,]{beamer}
\setbeamertemplate{caption}[numbered]
\setbeamertemplate{caption label separator}{: }
\setbeamercolor{caption name}{fg=normal text.fg}
\beamertemplatenavigationsymbolsempty
\usepackage{lmodern}
\usepackage{amssymb,amsmath}
\usepackage{ifxetex,ifluatex}
\usepackage{fixltx2e} % provides \textsubscript
\ifnum 0\ifxetex 1\fi\ifluatex 1\fi=0 % if pdftex
  \usepackage[T1]{fontenc}
  \usepackage[utf8]{inputenc}
\else % if luatex or xelatex
  \ifxetex
    \usepackage{mathspec}
  \else
    \usepackage{fontspec}
  \fi
  \defaultfontfeatures{Ligatures=TeX,Scale=MatchLowercase}
\fi
\usetheme[]{Montpellier}
\usecolortheme{dolphin}
\usefonttheme{structurebold}
% use upquote if available, for straight quotes in verbatim environments
\IfFileExists{upquote.sty}{\usepackage{upquote}}{}
% use microtype if available
\IfFileExists{microtype.sty}{%
\usepackage{microtype}
\UseMicrotypeSet[protrusion]{basicmath} % disable protrusion for tt fonts
}{}
\newif\ifbibliography
\hypersetup{
            pdftitle={Eléments d'analyse combinatoire},
            pdfauthor={Jacques van Helden},
            pdfborder={0 0 0},
            breaklinks=true}
\urlstyle{same}  % don't use monospace font for urls
\usepackage{longtable,booktabs}
\usepackage{caption}
% These lines are needed to make table captions work with longtable:
\makeatletter
\def\fnum@table{\tablename~\thetable}
\makeatother

% Prevent slide breaks in the middle of a paragraph:
\widowpenalties 1 10000
\raggedbottom

\AtBeginPart{
  \let\insertpartnumber\relax
  \let\partname\relax
  \frame{\partpage}
}
\AtBeginSection{
  \ifbibliography
  \else
    \let\insertsectionnumber\relax
    \let\sectionname\relax
    \frame{\sectionpage}
  \fi
}
\AtBeginSubsection{
  \let\insertsubsectionnumber\relax
  \let\subsectionname\relax
  \frame{\subsectionpage}
}

\setlength{\parindent}{0pt}
\setlength{\parskip}{6pt plus 2pt minus 1pt}
\setlength{\emergencystretch}{3em}  % prevent overfull lines
\providecommand{\tightlist}{%
  \setlength{\itemsep}{0pt}\setlength{\parskip}{0pt}}
\setcounter{secnumdepth}{0}

\title{Eléments d'analyse combinatoire}
\subtitle{Probabilités et statistique pour la biologie (STAT1)}
\author{Jacques van Helden}
\date{2018-11-09}

\begin{document}
\frame{\titlepage}

\begin{frame}
\tableofcontents[hideallsubsections]
\end{frame}

\section{Théorie}\label{theorie}

\begin{frame}{Arrangements}

On appelle \textbf{arrangements} les tirages \emph{ordonnés} effectués
\emph{sans remise} au sein d'un ensemble.

Nombre d'arrangements de \(x\) éléments tirés parmi \(n\).

\[A^x_n = \frac{n!}{(n - x)!} = \frac{n(n-1) \ldots (n-x +1) (n - x) (n-x-1) \ldots 2 \cdot 1}{(n - x) (n-x-1) \ldots 2 \cdot 1} = n \cdot (n-1) \cdot \ldots \cdot (n-x+1)\]

\textbf{Application typique:}
\textbf{\href{https://fr.wikipedia.org/wiki/Tierc\%C3\%A9_(jeu)}{tiercé}}
dans l'ordre. Les joueurs parient sur les trois chevaux gagnants d'une
course (\(x=3\)). Pour \(n=15\) chevaux partants, il existe
\(n \cdot (n-1) \cdot (n-2) = 15 \cdot 14 \cdot 13 = 2730\)
possibilités.

\end{frame}

\begin{frame}{Combinaisons}

On appelle \textbf{combinaisons} le nombre de sous-ensembles qu'on peut
tirer \emph{sans remise} dans un ensemble, sans tenir compte de l'ordre
des éléments tirés.

Ce nombre est fourni par le \textbf{coefficient binomial}.

\[\binom{n}{x} = C^x_n = \frac{n!}{x! (n-x)!}\]

\textbf{Attention: } les paramètres sont placés différemment dans la
première (\(binom{n}{x}\), ``x parmi n'') et la seconde notation
(\(C^x_n\), ``choose'').

\textbf{Applications typiques: }

\begin{itemize}
\item
  \textbf{\href{https://fr.wikipedia.org/wiki/Tierc\%C3\%A9_(jeu)}{tiercé}}
  dans le désordre.

  \[\binom{n}{x} = \binom{15}{3} = C^3_{15} = \frac{15!}{3! 12!} = 455\]
\item
  jeu de \textbf{\href{https://fr.wikipedia.org/wiki/Loto}{loto}} (ou
  lotto): chaque joueur dispose d'une grille avec 90 numéros, et doit en
  cocher 6. Nombre de possibilités:
  \[\binom{n}{x} = \binom{90}{6} = C^6_{90} = \frac{90!}{6! 84!} = 6.2261463\times 10^{8}\]
\end{itemize}

\end{frame}

\section{Résumé des concepts et
formules}\label{resume-des-concepts-et-formules}

\begin{frame}{Tirages avec / sans remise}

Il existe deux types classiques de tirage d'éléments au sein d'un
ensemble: avec ou sans remise.

\begin{enumerate}
\def\labelenumi{\arabic{enumi}.}
\item
  \textbf{Tirage sans remise}: chaque élément peut être tiré au plus une
  fois. Exemples:

  \begin{itemize}
  \tightlist
  \item
    Jeu de \href{https://fr.wikipedia.org/wiki/Loto}{loto} (ou lotto).
  \item
    Sélection aléatoire d'un ensemble de gènes dans un génome.
  \end{itemize}
\item
  Tirage \textbf{avec remise}: chaque élément peut être tiré zéro, une
  ou plusieurs fois. Exemples:

  \begin{itemize}
  \tightlist
  \item
    Jeu de dés. A chaque lancer on dispose des mêmes possibilités (6
    faces).
  \item
    Génération d'une séquence aléatoire, par sélection itérative d'un
    élément dans l'ensemble des résidus (4 nucléotides pour l'ADN, 20
    acides aminés pour les protéines).
  \end{itemize}
\end{enumerate}

\end{frame}

\begin{frame}{Formules}

\begin{longtable}[]{@{}llll@{}}
\toprule
\begin{minipage}[b]{0.11\columnwidth}\raggedright\strut
Remise\strut
\end{minipage} & \begin{minipage}[b]{0.10\columnwidth}\raggedright\strut
Ordre\strut
\end{minipage} & \begin{minipage}[b]{0.16\columnwidth}\raggedright\strut
Formule\strut
\end{minipage} & \begin{minipage}[b]{0.42\columnwidth}\raggedright\strut
Description\strut
\end{minipage}\tabularnewline
\midrule
\endhead
\begin{minipage}[t]{0.11\columnwidth}\raggedright\strut
Oui\strut
\end{minipage} & \begin{minipage}[t]{0.10\columnwidth}\raggedright\strut
Oui\strut
\end{minipage} & \begin{minipage}[t]{0.16\columnwidth}\raggedright\strut
\(n^x\)\strut
\end{minipage} & \begin{minipage}[t]{0.42\columnwidth}\raggedright\strut
\textbf{Exponentielle}: séquences de \(x\) éléments tirés dans un
ensemble de taille \(n\), avec remise.\strut
\end{minipage}\tabularnewline
\begin{minipage}[t]{0.11\columnwidth}\raggedright\strut
Non\strut
\end{minipage} & \begin{minipage}[t]{0.10\columnwidth}\raggedright\strut
Oui\strut
\end{minipage} & \begin{minipage}[t]{0.16\columnwidth}\raggedright\strut
\(n!\)\strut
\end{minipage} & \begin{minipage}[t]{0.42\columnwidth}\raggedright\strut
\textbf{Factorielle}: toutes les permutations d'un ensemble de taille
\(n\)\strut
\end{minipage}\tabularnewline
\begin{minipage}[t]{0.11\columnwidth}\raggedright\strut
Non\strut
\end{minipage} & \begin{minipage}[t]{0.10\columnwidth}\raggedright\strut
Oui\strut
\end{minipage} & \begin{minipage}[t]{0.16\columnwidth}\raggedright\strut
\(A^x_n = \frac{n!}{x!}\)\strut
\end{minipage} & \begin{minipage}[t]{0.42\columnwidth}\raggedright\strut
\textbf{Arrangements} : listes (ordonnée) de \(x\) éléments tirés dans
un ensemble de taille \(n\)\strut
\end{minipage}\tabularnewline
\begin{minipage}[t]{0.11\columnwidth}\raggedright\strut
Non\strut
\end{minipage} & \begin{minipage}[t]{0.10\columnwidth}\raggedright\strut
Non\strut
\end{minipage} & \begin{minipage}[t]{0.16\columnwidth}\raggedright\strut
\(C^x_n = \binom{n}{x} = \frac{n!}{x! (n - x) !}\)\strut
\end{minipage} & \begin{minipage}[t]{0.42\columnwidth}\raggedright\strut
\textbf{Combinaisons} : ensembles (non ordonnés) de \(x\) éléments tirés
dans un ensemble de taille \(n\)\strut
\end{minipage}\tabularnewline
\bottomrule
\end{longtable}

\end{frame}

\section{Exercices}\label{exercices}

\begin{frame}{Exercice: dénombrement d'oligomères}

L'ADN est composée de 4 nucléotides distincts dénotés par les lettres A,
C, G, T, et les protéines de 20 acides aminés.

Pour chacun de ces deux types de polymères, combien d'oligomères
distincts peut-on former en polymérisant 20 résidus (``20-mères'')~?

\textbf{Approche suggérée}: simplifiez le problème au maximum, en
commençant par des polymères beaucoup plus courts (1 résidu, 2 résidus).

\textbf{Questions subsidiaires}:

\begin{itemize}
\tightlist
\item
  Généralisez la formule pour les oligomères d'une longueur arbitraire
  \(k\) (\textbf{``k-mères''}).
\item
  Quel est le nom de la fonction donnant le résultat~?
\item
  Dans ce processus, quel est le mode de sélection des résidus:
  \textbf{avec ou sans remise}~?
\end{itemize}

\end{frame}

\begin{frame}{Solution: dénombrement d'oligomères}

\begin{itemize}
\item
  Il s'agit d'un tirage avec remise: à chaque position de la séquence on
  a le choix entre \(n\) résidus (4 pour les acides nucléiques, 20 pour
  les protéines).
\item
  Cas trivial: séquence d'un seul résidu \(\rightarrow\) le nombre de
  possibilités est \(n\).
\item
  Pour chacune des \(n\) possibilités de premier résidu, il y a \(n\)
  possibilités pour choisir le second résidu \(\rightarrow\) il existe
  \(n \cdot n = n^2\) séquences de taille 2.
\item
  Pour chacune d'entre elles, \(n\) résidus possibles en \(3^{ème}\)
  position \(\rightarrow\) il existe \(n^2 \cdot n = n^3\) séquences
  distinctes de taille 3.
\item
  En généralisant, il existe \(n^x\) séquences distinctes de taille
  \(x\).
\item
  Dans notre cas, la taille des séquences \(x=20\). On a donc

  \begin{itemize}
  \tightlist
  \item
    \(N = n^x = 4^20 = 1.1\times 10^{12}\) oligonucléotides distincts
  \item
    \(N = n^x = 20^20 = 1.05\times 10^{26}\) oligopeptides distincts
  \end{itemize}
\end{itemize}

\end{frame}

\begin{frame}{Exercice: oligomères sans résidus répétés}

Combien d'oligomères peut-on former (ADN ou peptides) en utilisant
chaque résidu une et une seule fois~?

\textbf{Approche suggérée}: agrégez progressivement les résidus, en vous
demandant à chaque étape combien d'entre eux n'ont pas encore été
incorporés.

\textbf{Questions subsidiaires}:

\begin{itemize}
\tightlist
\item
  Généralisez la formule pour des séquences d'objets tirés dans un
  ensemble de taille arbitraire (\(n\)).
\item
  Quel est le nom de la fonction donnant le résultat~?
\item
  Dans ce processus, quel est le mode de sélection des résidus:
  \textbf{avec ou sans remise}~?
\end{itemize}

\end{frame}

\begin{frame}{Solution: oligomères sans résidus répétés}

\begin{itemize}
\item
  Premier résidu: \(n\) possibilités.
\item
  Dès le moment où on a choisi ce premier résidu, il ne reste plus que
  \(n-1\) possibilités pour le second. On a donc \(n \cdot (n-1)\)
  possibilités pour les deux premiers résidus.
\item
  Pour la troisième position, en fonction des deux premiers résidus déjà
  tirés, il ne reste que \(n-2\) résidus. On a donc
  \(n \cdot (n-1) \cdot (n-2)\) possibilités pour les 3 premières
  positions de la séquence.
\item
  Par extension, le nombre total de possibilités est donc (en supposant
  \(n\) suffisamment grand)
  \[n! = n \cdot (n-1) \cdot \ldots \cdot 2 \cdot 1\].
\item
  Dans notre cas:

  \begin{itemize}
  \tightlist
  \item
    \(n! = 4! = 24\) oligonucléotides comportant exactement 1 fois
    chaque nucléotide (taille 4)
  \item
    \(n! = 20! = 2.43\times 10^{18}\) oligopeptides comportant
    exactement 1 fois chaque acide aminé (taille 20)
  \end{itemize}
\end{itemize}

\end{frame}

\begin{frame}{Exercice: listes (ordonnées) de gènes}

Lors d'en expérience de transcriptome indiquant le niveau d'expression
de tous les gènes de la levure. Sachant que le génome comporte 6000
gènes, combien de possibilité existe-t-il pour sélectionner les 15 gènes
les plus fortement exprimés (\textbf{en tenant compte} de l'ordre
relatif de ces 15 gènes)~?

\textbf{Approche suggérée}: comme précédemment, simplifiez le problème
en partant de la sélection minimale, et augmentez progressivement le
nombre de gènes (1 gène, 2 gènes, \ldots{}).

\textbf{Questions subsidiaires}:

\begin{itemize}
\tightlist
\item
  Trouvez un exemple familier de jeu de pari apparenté à ce problème.
\item
  Généralisez la formule pour la sélection d'une liste de \(x\) gènes
  dans un génome qui en comporte \(n\).
\end{itemize}

\end{frame}

\begin{frame}{Solution: listes (ordonnées) de gènes}

Il s'agit d'une sélection \textbf{sans remise} (chaque gène apparaît à
une et une seule position dans la liste de tous les gènes), et
\textbf{ordonnée} (les mêmes gènes pris dans un ordre différent sont
considérés comme un résultat différent).

\begin{itemize}
\tightlist
\item
  Pour le premier gène, il y a \(n=6000\) possibiité.
\item
  Dès le moment où on connaît le premier gène, il n'existe plus que 5999
  possibilités pour le second, et donc
  \(n \cdot (n-1) = 6000 \cdot 5999\) possibilités pour la suite des
  deux premiers gènes;
\item
  Par extension, il existe
  \(6000 \cdot 5999 \cdot 5998 \cdot \ldots \cdot 5986 = 4.62\times 10^{56}\)
  possibilités pour les 15 premiers gènes.
\item
  En généralisant à la liste des \(x\) premiers gènes dans un ensemble
  de \(n\), on obtient
  \(N = n \cdot (n-1) \cdot (n-2) \cdot ... \cdot (n-x+1)\).
\end{itemize}

\end{frame}

\begin{frame}{Exercice: ensembles (non-ordonnés) de gènes}

Lors d'une expérience de transcriptome indiquant le niveau d'expression
de tous les gènes de la levure. Sachant que le génome comporte 6000
gènes, combien de possibilité existe-t-il pour sélectionner les 15 gènes
les plus fortement exprimés (\textbf{sans tenir compte} de l'ordre
relatif de ces 15 gènes)~?

\textbf{Approche suggérée}: comme précédemment, simplifiez le problème
en partant de sélections minimales (1 gène, 2 gènes, \ldots{}) et
généralisez la formule.

\textbf{Questions subsidiaires}:

\begin{itemize}
\tightlist
\item
  Trouvez un exemple familier de jeu de pari apparenté à ce problème.
\item
  Généralisez la formule pour la sélection d'un ensemble de \(x\) gènes
  dans un génome qui en comporte \(n\).
\item
  Connaissez-vous le nom de la formule ainsi trouvée~?
\end{itemize}

\end{frame}

\begin{frame}{Solution: ensembles (non-ordonnés) de gènes}

\begin{itemize}
\tightlist
\item
  Pour une sélection d'un seul gène, il existe \(n=6000\) possibilité.
\item
  Pour 2 gènes, il existe \(n \cdot (n-1) = 6000 \cdot 5999\)
  arrangements, mais ceci inclut deux fois chaque paire de gènes
  (\((a, b)\) et \((b, a)\)). Le nombre d'ensembles non ordonnés est
  donc \(N = n \dot (n-1)/2\).
\item
  De même, pour 3 gènes, il faut diviser le nombre d'arrangements
  (\(A^x_n = \frac{n!}{(n-x)!} = 6000 \cdot 5999 \cdot 5998\)) par le
  nombre de permutations parmi tous les triplets de gènes
  (\((a, b, c), (a, c, b), (b, a, c) \ldots\)), ce qui donne
  \(\frac{6000!}{(6000-3)! 3!} = \frac{6000 \cdot 5999 \cdot 5888}{6} = 3.6\times 10^{10}\).
\item
  Pour 15 gènes, on obtient
  \(\frac{n!}{(n-x)!x!} = \frac{6000!}{5985! \cdot 3!} = 3.53\times 10^{44}\)
  \emph{combinaisons} possibles.
\end{itemize}

\end{frame}

\begin{frame}{Exercice~: mutagénèse}

\emph{On soumet un fragment d'ADN de 1 kilobase à un traitement mutagène
qui provoque des mutations ponctuelles (substitutions) à 5 positions
distinctes indépendantes. Combien de séquences possibles existe-t-il
pour le fragment muté~?}

\end{frame}

\begin{frame}{Solution exercice 1: mutagenèse}

\emph{On soumet un fragment d'ADN de 1 kilobase à un traitement mutagène
qui provoque des mutations ponctuelles (substitutions) à 5 positions
distinctes indépendantes. Combien de séquences possibles existe-t-il
pour le fragment muté~?}

Il s'agit de choisir au hasard 5 positions mutantes parmi les 1000
nucléotides du fragment d'ADN. Il s'agit d'un choix sans remise (chaque
position ne peut être tirée qu'une fois), on choisit donc le coefficient
binomial.

\[\binom{n}{x} = \binom{1000}{5} = C^5_{1000} = \frac{1000!}{5! 995!} = 8.2502913\times 10^{12}\]

\end{frame}

\begin{frame}{Exercice~: oligopeptides \(3 \times 20\)}

\emph{Combien d'oligopeptides de taille 60 peut-on former en utilisant
exactement 3 fois chaque acide aminé~?}

\end{frame}

\begin{frame}[fragile]{Solution de l'exercice~: oligopeptides
\(3 \times 20\)}

\emph{Combien d'oligopeptides de taille 60 peut-on former en utilisant
exactement 3 fois chaque acide aminé~?}

Commençons par générer une séquence particulière qui remplit ces
conditions, en concaténant 3 copies de chaque acide aminé, dans l'ordre
alphabétique.

\begin{verbatim}
AAACCCDDDEEEFFFGGGHHHIIIKKKLLLMMMNNNPPPQQQRRRSSSTTTVVVWWWYYY
\end{verbatim}

Toutes les permutations de ces 60 lettres sont des solutions valides. En
voici trois exemples.

\begin{verbatim}
FSWYRPDIVMMDYSWDKHKTEKGHVCQNNAFLHYNEAQCMPQGILPCTTFAIVWLGRRSE
\end{verbatim}

\begin{verbatim}
WHAFSEDTPHCYEGAMMAMIQWHVYVDLKRWFPQCCREIVLYTIQGRNLGTNFPKSNDSK
\end{verbatim}

\begin{verbatim}
YLCPQMEWVMDAAKGETQSICVRIQHFGGPTWHLDLYFRSFYIHCPNNTKVSRNDWMEKA
\end{verbatim}

Cependant, il faut prendre en compte le fait que certaines permutations
sont identiques (toutes celles où l'on permute deux acides aminés
identiques). La difficulté de l'exercice sera donc de dénombrer le
nombre de permutations \emph{distinctes}.

\end{frame}

\end{document}
